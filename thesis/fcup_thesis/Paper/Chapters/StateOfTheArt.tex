\chapter{State of the art}
\label{chapters:state_of_the_art}

\section{Author name (dis)ambiguity}
\label{chapters:sections:author_name_ambiguity}
Every day, millions of scientific papers are published, and the number of authors is increasing. This growth has led to a significant increase in the number of authors with the same name. This phenomenon is known as author name ambiguity. Author name ambiguity is a problem that affects the quality of bibliographic databases, citation analysis, and information retrieval systems. It is a challenge for researchers, publishers, and librarians to identify and disambiguate authors with the same name. Author name ambiguity can lead to incorrect attribution of authorship, misinterpretation of research results, and duplication of research efforts. It can also affect the accuracy of citation analysis and information retrieval systems. Author name ambiguity is a significant problem in the scientific community, and it is essential to address it to ensure the quality and integrity of scientific research.
\break
There are two types of ambiguity that can relate to author names: 

\begin{itemize}
	\item \textbf{Homonyms}: Homonyms are authors with the same name but different identities
	\item \textbf{Synonyms}: Synonyms are authors with different names but the same identity
\end{itemize}

The problem of author name ambiguity lies in distinguishing different authors that have the same name and identifying the same author that has different names.

\subsection{Definition}
Assuming that there are $\tau$ distinct authors in a database A such that: 
\begin{displaymath}
	A = \{ a_1, a_2, ..., a_\tau \} 
\end{displaymath}
And, there is also a database $P$, with $\kappa$ distinct papers:
\begin{displaymath}
	P = \{ p_1, p_2, ..., p_\kappa \}
\end{displaymath}
With these, we can define also a set of papers $J \subseteq P$ for each author $a_i$ like so:
\begin{displaymath}
	X = (a_i, J) \quad a_i \in A
\end{displaymath}

Test