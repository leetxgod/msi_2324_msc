\chapter{Counting}\label{chap:counting}
In this chapter, we will explore the concept of counting in the context of formal languages and automata theory as well as explain an attempt that was made towards match counting using a partial derivative automaton construction and why it didn't work.

\section{Counting in Formal Languages}
In formal language theory, counting refers to the ability of a language or an automaton to enforce numeric constraints over the number of symbols or patterns within strings. Specifically, it deals with the ability to recognize whether certain elements occur a specified number of times—or in a specific numerical relationship to others.
Current tools are already able to do this, including some non-backtracking matchers.

\section{Derivatives of operations in Extended Regular Expressions}
In order to be able to match using counting regular expressions, one must first solve the derivatives for fixed and bounded repetition.

\subsection{Fixed Repetition}
Given $r = ab$ such that $r^2 = abab$, we have:
\begin{align*}
	D_a(r^2) &= D_a(r \cdot r) \\
	&= D_a(abab)  \\
	&= D_a(a)bab + D_a(bab) \\
	&= \varepsilon bab + \emptyset \\
	&= bab
\end{align*}


\subsection{Bounded Repetition}
Given $r = ab$ such that $r^2 = abab$ and $r^3 = ababab$, we have:
\begin{align*}
	D_a(r^{[2,4[}) &= D_a(abab) + D_a(ababab) \\
	&= D_a(abab) + D_a(ababab) \\
	&= D_a(a)bab + D_a(bab) + D_a(a)babab + D_a(babab) \\
	&= \varepsilon bab + \emptyset + babab + \emptyset \\
	&= bab + babab
\end{align*}

\section{Extended Regular Expressions in FAdo}
Recalling back from \ref{chap:prelim:extended_re}, in order to support extended regular expressions, one needs to extend the base class for unary operations in FAdo