\chapter{Counting}\label{chap:counting}
In this chapter, we will explore the concept of counting in the context of formal languages and automata theory as well as explain the implementation for fixed and bounded repetition in FAdo.

\section{Counting in Formal Languages}
In formal language theory, counting refers to the ability of a language or an automaton to enforce numeric constraints over the number of symbols or patterns within strings. Specifically, it deals with the ability to recognize whether certain elements occur a specified number of times—or in a specific numerical relationship to others.
Current tools are already able to do this, including some non-backtracking matchers.

\section{Derivatives of operations in Extended Regular Expressions}
Given a word $w = xyz$ and a regular expression $\alpha$, $w \in L(\alpha)$ if $\varepsilon(D_w(\alpha)) = \varepsilon$.
In order to match with fixed and bounded repetition, one must first solve the derivatives for those operations.

%So, in order to match regular expressions, one must first solve the derivatives for fixed and bounded repetition.

\break

\subsection{Fixed Repetition}
For a fixed repetition, we need $r$ to occur $n$ times, resulting in the notion $r^n$.

%\noindent Given $r = ab$ such that $r^2 = abab$, we have:
%\begin{align*}
%	D_a(r^2) &= D_a(r \cdot r) \\
%	&= D_a(abab)  \\
%	&= D_a(a)bab + D_a(bab) \\
%	&= \varepsilon bab + \emptyset \\
%	&= bab
%\end{align*}

% PROOF
\noindent Given $e \in RegExp(\Sigma)$, we want to show that
\begin{align*}
	D_a(e^n) &= D_a(e)e^{n-1}, \quad n>1.
\end{align*}

\noindent It is true that
\begin{align*}
	D_a(e) = D_a(e)e^0 = D_a(e). 
\end{align*}

\noindent We can also suppose that
\begin{align*}
	D_a(e^{n-1}) = D_a(e)e^{n-2}.
\end{align*}

\noindent We have that
\begin{align*}
	D_a(e^n) &= D_a(ee^{n-1}) \\
	&= D_a(e)e^{n-1} + \varepsilon(e)D_a(e^{n-1}) \\
	&= D_a(e)e^{n-1} + \varepsilon(e)D_a(e)e^{n-2} \\
	&= (D_a(e)e + \varepsilon(e)D_a(e))e^{n-2} \\
	&= (D_a(e)(e+\varepsilon(e)))e^{n-2} \\
	&= D_a(e)e^{n-1}
\end{align*}

% ADD e^n results for n=0, n=1 and n>1

\subsection{Bounded Repetition}
Supplementing the fixed repetition notion, we now add a maximum bound so that $r$ can occur at least $n$ times and at maximum $m-1$ times: $r^{n,m}$.

Given $r = ab$ such that $r^2 = abab$ and $r^3 = ababab$, we have:
\begin{align*}
	D_a(r^{[2,4[}) &= D_a(abab) + D_a(ababab) \\
	&= D_a(abab) + D_a(ababab) \\
	&= D_a(a)bab + D_a(bab) + D_a(a)babab + D_a(babab) \\
	&= \varepsilon bab + \emptyset + babab + \emptyset \\
	&= bab + babab
\end{align*}

This also extends for $r^{[m, \inf[}$, which is the same as having $r^m \cdot r^*$:

\begin{align*}
	D_a(r^{[2,\inf[}) &= D_a(abab) + D_a((ab)^*) \\
	&= D_a(a)bab + D_a(bab) + D_a(ab)\cdot(ab)^* \\
	&= \varepsilon bab + \emptyset + (D_a(a)b + D_a(b)) \cdot (ab)^* \\
	&= bab + (\varepsilon b + \emptyset) \cdot (ab)^* \\
	&= bab + b(ab)^*
\end{align*}

\noindent Given $e \in RegExp(\Sigma)$, $n>=1$ and $m>n$ we want to show that
\begin{align*}
	D_a(e^{n,m}) &= \sum_{i=n}^{m} D_a(e^i) \\
	&= \sum_{i=n}^{m} (D_a(e))e^{i-1} \\
	&= D_a(e) \sum_{j=n-1}^{m-1} e^j \\
	&= D_a(e) e^{n-1,m-1}
\end{align*}

\noindent So, for $n=1$ we have that
\begin{align*}
	D_a(e^{n,m}) &= D_a(e)e^{0,m-1} \\
	&= D_a(e) + (D_a(e))e^{1,m-1}
\end{align*}

\noindent And, for $n>=1$ we have that
\begin{align*}
	D_a(e^{n,m}) &= (D_a(e))e^{n-1,m-1}
\end{align*}
%For instance, if $m = 2$, we have that $r^{[2, \inf[}$

The regular expression $r^{[2, \inf[}$ will match any $(ab)^k$ with $k \geq 2$.

%Given $w = abababab$ and $m = 2$,  will match 

%The string $abababab$ can be matched by $r^{[2, \inf[}$


\section{Class implementation in FAdo}
In order to implement these new syntactic constructs of extended regular expressions in FAdo, we added exact-power and counted-repetition nodes to the regular expression abstract syntax tree and define their derivatives, partial derivatives and linear form cases.

\subsection{CPower}
\textit{CPower} is the class responsible for the construction of regular expressions using the fixed repetition operator.

\begin{lstlisting}[language=Python]
class CPower(Unary):
	def __init__(self, arg, n, sigma=None):
	self.arg = arg
	self.n = n
	self.Sigma = sigma
	self._ewp = False if self.n > 0 else True
	
	...
\end{lstlisting}

As shown above, the class inherits from the \textit{Unary}, which only defines the \textit{Unary.Sigma} and \textit{Unary.arg} properties, letting the class hold a alphabet set (representing $\Sigma$) and the symbol used to construct this operator.

In order to integrate the class fully with \textit{FAdo}, some more methods had to be implemented.

\begin{lstlisting}[language=Python]
	def linearForm(self):
		arg_lf = self.arg.linearForm()
		lf = dict()
		for head in arg_lf:
			lf[head] = set()
			for tail in arg_lf[head]:
				if self.n == 0:
					lf[head].add(CEmptySet(self.Sigma))
				elif self.n == 1:
					lf[head].add(CEpsilon(self.Sigma))
				else:
					lf[head].add(CPower(self.arg, self.n-1, self.Sigma))
	
		return lf

	def partialDerivatives(self, sigma):
		return self.arg.partialDerivatives(sigma)
	
	def derivative(self, sigma):
		if str(sigma) in str(self.arg):
			if self.n == 0:
				return CEmptySet(sigma)
			elif self.n == 1:
				return self.arg.derivative(sigma)
			else:
				return CConcat(self.arg.derivative(sigma), CPower(self.arg, self.n-1, self.Sigma))
		else:
			return CEmptySet(sigma)
\end{lstlisting}



\section{Extended Regular Expressions in FAdo}
Recalling back from \ref{chap:prelim:extended_re}, in order to support extended regular expressions, one needs to extend the base class for unary operations in FAdo