%%%%%%%%%%%%%%%%%%%%%%%%%%%%%%%%%%%%%%%%%%%%%%%%%%%%%%%%%%%%
% Pedro Brandao's trial to get a template for thesis for students
% Used the upthesis from Fernando Silva (see upthesis).
% See also the packages file.
% 2014/07/07 First draft
% 2014/07/21 
%  pbrandao: added the list of listings (it should produce portuguese name if
%           babel is set to portuguese, see packages.tex). Changed usepackage of babel
%           to be before input packages.tex to allow test

%
%%%%%%%%%%%%%%%%%%%%%%%%%%%%%%%%%%%%%%%%%%%%%%%%%%%%%%%%%%%%
% LTeX: language=portuguese

% makes all pages the height of the text on that page. No extra vertical space is added.
\raggedbottom 
% setting it to report will remove the blank pages before each chapter
\documentclass[11pt,a4paper,twoside]{book}


%%%%%%%%%%%%%%%%%%%%%%%%%%%%%%%%%%%%%%%%%%%%%%%%%%%%%%%%%%%%
%%%   Packages that need to be configured for the thesis
%%%%%%%%%%%%%%%%%%%%%%%%%%%%%%%%%%%%%%%%%%%%%%%%%%%%%%%%%%%%

% Language settings
% use UKenglish for UK or leave blank for US English
% it will also change the names for some of the chapters (list of tables, figures, content,
%\usepackage[UKenglish]{babel}
%\usepackage[portuguese]{babel}
% If you have multiple languages (as in this text) you can use both, leaving last the main one.
% Then you can use 
% \begin{otherlanguage}{UKenglish}
% text in other language, UKEnglish in this case
% \end{otherlanguage}
\usepackage[UKenglish]{babel}


%%%%%%%%%%%%%%%%%%%%%%%%%%%%%%%%%%%%%%%%%%%%%%%%%%%%%%%%%%%%
%%%   Packages uses language definitions
% see file below for more packages and settings
%%%%%%%%%%%%%%%%%%%%%%%%%%%%%%%%%%%%%%%%%%%%%%%%%%%%%%%%%%%%
\usepackage{upthesis}
\usepackage{xcolor}

% the bibliography file
\addbibresource{refs.bib}


%% Definitions for acronyms
\newcommand*\acronymbiggest{HTTP} % defines the biggest Acronym used to define the column size



\usepackage[
%backref={section},
%pagebackref, % for getting references to the page where the citation is (in the biblio), note that biblatex above also has this
pdfpagelabels=false
]{hyperref}
\hypersetup{pdftitle={Test}, %nao suporta acentos
   pdfkeywords ={palavras chave},
   pdfsubject = {assunto},
	bookmarksnumbered=true,
   pdfauthor ={Your name}, % see other options on manual (can be page) needs empty line on bibitem
   plainpages=false, 
   pdfborder={0 0 0},
   colorlinks,%colorlinks=false,
   breaklinks=true,
	%hyperindex=true 	% Makes the page numbers of index entries into hyperlinks. Relays on unique page anchors (pageanchor) 
% see for colors http://mirror.ctan.org/macros/latex/contrib/xcolor/xcolor.pdf
   linkcolor=	Sepia, %MidnightBlue,% BlueViolet,%Sepia, % Color for normal internal links.
   %anchorcolor=black,% Color for anchor text.
   citecolor=RedViolet,% Color for bibliographical citations in text.
   %filecolor=cyan% Color for URLs which open local files.
   %menucolor=red% Color for Acrobat menu items.
   %runcolor=filecolor% Color for run links (launch annotations).
   urlcolor=NavyBlue% Color for linked URLs.
}
%use the same style for \url as the text
% from http://en.wikibooks.org/wiki/LaTeX/Hyperlinks#Customization
\urlstyle{same}


\begin{document}

\title{Thesis title}
%% The following is not currently being used.
% Part of the coverp in upthesis.sty
% \submitionplace{Tese submetida à Faculdade de Ciências da \\
%   Universidade do Porto para obtenção do grau de Mestre \\ 
%   em Ciência de Computadores}
% \author{Nome do autor}
% \department{Departamento de Ciência de Computadores \\ Faculdade de
%   Ciências da Universidade do Porto}
% \submitdate{Setembro 2015}


\beforepreface%



\prefacesection{Abstract}

Hey, this is the abstract of my thesis. It should be a brief summary of the work, highlighting the main objectives, methods, results, and conclusions. The abstract should be concise and informative, allowing readers to quickly understand the essence of the research.
\textbf{Keywords:} key, word.


\prefacesection{Resumo}
O teu resumo COOL, its me TEST WOWIEESSS

\textbf{Palavras-chave:} palavra, chave..


\prefacesection{Acknowledgements}

First of all, I would like to thank my family, etc, etc
\dedicationpage{Dedico à minha mãe \ldots}



% end of thesis preamble
\afterpreface%

%% main tex here
%% By putting the chapter names here, one can just comment the content in the chapters 
%% and produce a pdf with the correct chapter number.
%% If you want further configurability you can use subfiles package
%% https://www.ctan.org/pkg/subfiles

%\chapter{Introduction}\label{chap:intro}
\chapter{Introduction}\label{chap:intro}

%https://www.usenix.org/system/files/sec22-turonova.pdf
%https://arxiv.org/abs/2407.20479
%https://www.regular-expressions.info/catastrophic.html

In this chapter, the problem is overviewed, the study's importance is explained along with goals for the proposed solution. 

\section{Background}
Regular expressions are a foundational tool in computer science, widely used in pattern matching, lexical analysis, input validation, and string processing. Their expressiveness and concise syntax make them a powerful language for describing regular languages.

% However, when implemented without care, especially in backtracking-based engines, regular expressions can become a source of serious security vulnerabilities.

A regular expression $R$ is used (along with an input $W$) in regex matching engines. The matching engines will verify if $W$ is fully matched by $R$, meaning that the entire input is a match - or they will verify if a substring of $W$ is matched by $R$. 

% Current matching engines commonly fall into one of the following categories:
% \begin{itemize}
% 	\item \textbf{Finite State Machine Regular Expression Engines}: A finite state machine (or finite \emph{automaton}) is built and evaluated using every symbol $\sigma \in W$. Used \emph{UNIX}-based systems.
% 	\item \textbf{Backtracking Regular Expression Engines}: Instead of building a finite state machine, 
% \end{itemize}

\section{Regular Expression Denial of Service}
One such vulnerability is known as \emph{Regular Expression Denial of Service} (ReDoS). ReDoS exploits the pathological worst-case behavior of certain regular expressions, causing exponential time complexity during matching. In typical backtracking matchers---such as those found in JavaScript, Java, and many scripting environments---ambiguous or nested expressions (especially involving repetition, such as \texttt{(a+)+}) can lead the engine to explore an exponential number of paths for certain crafted inputs. This behavior allows an attacker to intentionally supply inputs that force excessive computation, effectively rendering a service unavailable or degraded.

The root of the ReDoS problem lies not in regular expressions as a theoretical model but in how they are operationalized in software. While deterministic finite automata (DFAs) evaluate regular expressions in linear time, many real-world engines opt for backtracking NFAs due to their flexibility and ease of implementation. Unfortunately, these NFAs are susceptible to exponential blow-up in ambiguous or unguarded patterns.

\section{}

%\chapter{State of the Art}\label{chap:art}
\chapter{State of the Art}\label{chap:art}

\section{Overview of XYZ}
%%%%%%%%%%%%%%%%%%%%%%%%%%%%%%%%%%%%%%%%
Computers are devices that



%\chapter{Methodology}\label{chap:meth}
% \chapter{Methodology}\label{chap:meth}

The methodology chapter outlines the systematic approach taken to develop ...
Test
\section{Requirements Analysis}

The purpose of this requirements analysis is to describe in detail 

%\chapter{Desenho da Arquitetura}\label{chap:syst}
\chapter{Matching}\label{chap:devel}

The implementation chapter gives insights into 

\section{Normal Matching}
\subsection{Standard Matching}

\subsection{Greedy Matching}


\section{Multi Matching}



%\chapter{Resultados e análise}\label{chap:results}
% !TEX root = thesis.tex

\chapter{Results and Discussion}\label{chap:results}

This is a test

\section{Evaluation}

The methods of evaluating
 

%\chapter{Conclusões}\label{chap:conc}
\chapter{Conclusion}
In this work, we contribute with a new approach to address the problem of \ac{ReDoS}, focusing on a solution outside of traditional backtracking engines.
We explored the underlying causes of ReDoS vulnerabilities, particularly how certain finite automaton evaluators can lead to catastrophic performance, due to the deviation of the path of theory. To tackle this, we extended \textit{FAdo} to support fixed and bounded repetition operators. To recognize this new grammar, the default Lark grammar present in \textit{FAdo} was modified. The new type of automaton is capable of performing multiple overlapping matching.
Overall, this work contributes towards safer and more predictable regular expression evaluation by combining formal methods and a new matching strategy.

Future work may turn towards making these algorithms more efficient by switching to a different language or integrating them into a larger ecosystem. It would also be very interesting to develop a tool that generates overlapped text datasets

%With this, a new type of automaton was implemented and used to perform multiple overlapping matching.



%% references
%\renewcommand{\bibname}{Referências} % o babel portuguese coloca Bibliografia
% os meses do ficheiro bib poderão aparecer em inglês, caso se pretenda deve-se colocar o texto em português explicitamente no ficheiro bid
\cleardoublepage%
\phantomsection%
\printbibliography[heading=bibintoc]%


%% appendix
\appendix
\include{appendix1}

%% bye
\end{document}
