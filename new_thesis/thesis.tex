%%%%%%%%%%%%%%%%%%%%%%%%%%%%%%%%%%%%%%%%%%%%%%%%%%%%%%%%%%%%
% Pedro Brandao's trial to get a template for thesis for students
% Used the upthesis from Fernando Silva (see upthesis).
% See also the packages file.
% 2014/07/07 First draft
% 2014/07/21 
%  pbrandao: added the list of listings (it should produce portuguese name if
%           babel is set to portuguese, see packages.tex). Changed usepackage of babel
%           to be before input packages.tex to allow test

%
%%%%%%%%%%%%%%%%%%%%%%%%%%%%%%%%%%%%%%%%%%%%%%%%%%%%%%%%%%%%
% LTeX: language=portuguese

% makes all pages the height of the text on that page. No extra vertical space is added.
\raggedbottom 
% setting it to report will remove the blank pages before each chapter
\documentclass[11pt,a4paper,twoside]{book}


%%%%%%%%%%%%%%%%%%%%%%%%%%%%%%%%%%%%%%%%%%%%%%%%%%%%%%%%%%%%
%%%   Packages that need to be configured for the thesis
%%%%%%%%%%%%%%%%%%%%%%%%%%%%%%%%%%%%%%%%%%%%%%%%%%%%%%%%%%%%

% Language settings
% use UKenglish for UK or leave blank for US English
% it will also change the names for some of the chapters (list of tables, figures, content,
%\usepackage[UKenglish]{babel}
%\usepackage[portuguese]{babel}
% If you have multiple languages (as in this text) you can use both, leaving last the main one.
% Then you can use 
% \begin{otherlanguage}{UKenglish}
% text in other language, UKEnglish in this case
% \end{otherlanguage}
\usepackage[UKenglish]{babel}


%%%%%%%%%%%%%%%%%%%%%%%%%%%%%%%%%%%%%%%%%%%%%%%%%%%%%%%%%%%%
%%%   Packages uses language definitions
% see file below for more packages and settings
%%%%%%%%%%%%%%%%%%%%%%%%%%%%%%%%%%%%%%%%%%%%%%%%%%%%%%%%%%%%
\usepackage{upthesis}
\usepackage{xcolor}

% the bibliography file
\addbibresource{refs.bib}


%% Definitions for acronyms
\newcommand*\acronymbiggest{HTTP} % defines the biggest Acronym used to define the column size



\usepackage[
%backref={section},
%pagebackref, % for getting references to the page where the citation is (in the biblio), note that biblatex above also has this
pdfpagelabels=false
]{hyperref}
\hypersetup{pdftitle={Test}, %nao suporta acentos
   pdfkeywords ={palavras chave},
   pdfsubject = {assunto},
	bookmarksnumbered=true,
   pdfauthor ={Your name}, % see other options on manual (can be page) needs empty line on bibitem
   plainpages=false, 
   pdfborder={0 0 0},
   colorlinks,%colorlinks=false,
   breaklinks=true,
	%hyperindex=true 	% Makes the page numbers of index entries into hyperlinks. Relays on unique page anchors (pageanchor) 
% see for colors http://mirror.ctan.org/macros/latex/contrib/xcolor/xcolor.pdf
   linkcolor=	Sepia, %MidnightBlue,% BlueViolet,%Sepia, % Color for normal internal links.
   %anchorcolor=black,% Color for anchor text.
   citecolor=RedViolet,% Color for bibliographical citations in text.
   %filecolor=cyan% Color for URLs which open local files.
   %menucolor=red% Color for Acrobat menu items.
   %runcolor=filecolor% Color for run links (launch annotations).
   urlcolor=NavyBlue% Color for linked URLs.
}
%use the same style for \url as the text
% from http://en.wikibooks.org/wiki/LaTeX/Hyperlinks#Customization
\urlstyle{same}


\begin{document}

\title{Thesis title}
%% The following is not currently being used.
% Part of the coverp in upthesis.sty
% \submitionplace{Tese submetida à Faculdade de Ciências da \\
%   Universidade do Porto para obtenção do grau de Mestre \\ 
%   em Ciência de Computadores}
% \author{Nome do autor}
% \department{Departamento de Ciência de Computadores \\ Faculdade de
%   Ciências da Universidade do Porto}
% \submitdate{Setembro 2015}


\beforepreface%



% \prefacesection{Abstract}

% A long, long time ago... 

% This section should summarize the content of the dissertation, namely: explain the context of problem, describe the problem itself and address the work done to mitigate/solve the problem. Results and/or contributions should be mentioned.

% \prefacesection{Resumo}
% Há muito, muito tempo

% See the Abstract.
% Test for real
% Test

% \prefacesection{Agradecimentos}

% Obrigado a todos, obrigado \ldots

\prefacesection{Acknowledgements}
First and foremost, I want to thank professors Nelma Moreira and Rogério Reis, who have assisted me this last year. Their help, guidance, willingness and friendliness were fundamental for this work and me.
Thankfulness will never be enough to describe what I feel for my friends and family, who have helped me carry my woes and solaces along this journey. I am especially grateful to my mom and dad for taking care of me.
Lastly, I would also like to thank my work colleagues, who have been very thoughtful and patient with me along the way. 

\prefacesection{Abstract}
%Regular expressions (\emph{regex}) formalize a class of patterns definable over strings, corresponding to the family of regular languages in automata theory. They provide a declarative mechanism for specifying sets of strings, making them central both to theoretical models of computation and to practical applications such as string matching, input validation, and text parsing. Despite their utility, improperly constructed regular expressions can introduce serious security vulnerabilities. One of the most critical threats is the \ac{ReDoS} attack, in which carefully crafted inputs cause the regex engine to perform excessive and redundant processing. This results in dramatic slowdowns or even complete unresponsiveness of the system. \ac{ReDoS} poses a significant risk to web applications, APIs, and other input-facing systems, where user-controlled input is matched against vulnerable patterns.
\noindent Regular expressions (\emph{regex}) formalize a class of patterns definable over strings, corresponding to the family of regular languages in automata theory. They provide a declarative mechanism for specifying sets of strings, making them central both to theoretical models of computation and to practical applications such as string matching, input validation, and text parsing. Despite their utility, certain implementations of regular expression engines (particularly those based on backtracking) can introduce serious security vulnerabilities when combined with specific patterns. One of the most critical threats is the \ac{ReDoS} attack, in which carefully crafted inputs cause the regex engine to perform excessive and redundant processing. This results in dramatic slowdowns or even complete unresponsiveness of the system. \ac{ReDoS} poses a significant risk to web applications, APIs, and other input-facing systems, where user-controlled input is matched against vulnerable patterns.

\noindent In this work, bounded quantifiers (counting) were implemented in \textit{FAdo}. With this, we also propose a system to address \ac{ReDoS} by transforming regular expressions into a modified position automaton, a \ac{NFA} that tracks the exact start and end positions of all matches within an input string. This structure enables a matching function that computes all match positions, including overlapping ones, without relying on backtracking. By exhaustively and efficiently exploring the automaton's transitions, our approach avoids the exponential blowup typical of vulnerable engines, while preserving a somewhat full regex expressiveness.

\noindent We also review existing solutions present in state-of-the-art programming languages and libraries, such as \emph{RE\#} \cite{resharp_tool_paper} and \emph{Hyperscan} \cite{hyperscan_paper}.

\textbf{Keywords:} regular expressions, ReDoS, position automata, nondeterministic finite automata, pattern matching.

\prefacesection{Resumo}
\noindent As expressões regulares (\emph{regex}) formalizam uma classe de padrões definíveis sobre cadeias de caracteres, correspondendo à família das linguagens regulares na teoria dos autómatos. Fornecem um mecanismo declarativo para especificar conjuntos de cadeias, sendo, por isso, centrais tanto para modelos teóricos de computação como para aplicações práticas, tais como procura de padrões, validação de dados e análise de texto. Apesar da sua utilidade, certas implementações de motores de expressões regulares (particularmente aqueles baseados em \textit{backtracking}) podem introduzir vulnerabilidades de segurança graves quando utilizam padrões específicos. Uma das ameaças mais críticas é o ataque \ac{ReDoS}, onde certas entradas levam o motor de regex a realizar processamento excessivo e redundante. Isto resulta em lentidões significativas ou mesmo na completa inoperacionalidade do sistema. O \ac{ReDoS} representa um risco significativo para aplicações web, APIs e outros sistemas voltados para entrada de dados, onde a entrada controlada pelo utilizador é comparada com padrões vulneráveis.

\noindent Neste trabalho, foram implementados quantificadores limitados (contagem) no \textit{FAdo}. Com isto, propomos também um sistema para mitigar o \ac{ReDoS}, ao transformar expressões regulares num autómato de posições modificado --- um \ac{NFA} que regista exatamente as posições de início e fim de todas as ocorrências dentro de uma cadeia de entrada. Esta estrutura permite que uma função de \textit{matching} calcule todas as posições de ocorrência, incluindo as sobrepostas, sem recorrer a \textit{backtracking}. Ao explorar de forma exaustiva e eficiente as transições do autómato, a nossa abordagem evita a explosão exponencial típica de motores vulneráveis, mantendo, ao mesmo tempo, uma expressividade relativamente completa das expressões regulares.

\noindent Analisamos também soluções existentes em linguagens e bibliotecas de programação de ponta, como o \emph{RE\#} \cite{resharp_tool_paper} e o \emph{Hyperscan} \cite{hyperscan_paper}.

\textbf{Palavras-chave:} expressões regulares, ReDoS, autómatos de posições, autómatos finitos não deterministas, correspondência de padrões.
\dedicationpage{Dedico à minha mãe \ldots}



% end of thesis preamble
\afterpreface%

%% main tex here
%% By putting the chapter names here, one can just comment the content in the chapters 
%% and produce a pdf with the correct chapter number.
%% If you want further configurability you can use subfiles package
%% https://www.ctan.org/pkg/subfiles

%\chapter{Introduction}\label{chap:intro}
\chapter{Introduction}\label{chap:intro}

In this chapter, the problem is overviewed, the study’s importance is explained along with goals for the proposed solution. 

\section{Background}
Despite recent advances in~\cite{cloud}, .....

%\chapter{State of the Art}\label{chap:art}
\chapter{State of the Art}\label{chap:art}
This chapter sets the foundation for understanding the current landscape on \ldots

\section{Overview of XYZ}
%%%%%%%%%%%%%%%%%%%%%%%%%%%%%%%%%%%%%%%%
Computers are devices that



%\chapter{Methodology}\label{chap:meth}
\chapter{Methodology}\label{chap:meth}

The methodology chapter outlines the systematic approach taken to develop ...
Test
\section{Requirements Analysis}

The purpose of this requirements analysis is to describe in detail 

%\chapter{Desenho da Arquitetura}\label{chap:syst}

\chapter{Implementation}\label{chap:devel}

The implementation chapter gives insights into 

\section{Client-Server Architecture}

This section describes the client-server architecture, which is important in the development of the application. It focuses on coordination of the mobile/web clients developed using Flutter and the Firebase server to support instant data flow, secure sign-in, and retrieval/storing of the data.

\begin{lstlisting}[
  %language=yaml,                   % Specify language for syntax highlighting
  numbers=left,                    % Add line numbers on the left
  numberstyle=\tiny\color{gray},   % Style line numbers (smaller, gray)
  basicstyle=\footnotesize,         % Set code font size (smaller)
  caption=Flutter Project Dependencies,        % Add a caption
  frame=single,                    % Add a single frame around the listing
  showtabs=false,                   % Hide tabs (optional)
  breaklines=true,                   % Allow line breaks (optional)
  breakatwhitespace=false           % Prevent breaks within words (optional)
]
dependencies:
  flutter:
    sdk: flutter
  firebase_core: latest_version
  firebase_auth: latest_version
  cloud_firestore: latest_version
\end{lstlisting}


%\chapter{Resultados e análise}\label{chap:results}
% !TEX root = thesis.tex

\chapter{Results and Discussion}\label{chap:results}

This is a test

\section{Evaluation}

The methods of evaluating
 

%\chapter{Conclusões}\label{chap:conc}
\chapter{Conclusões}\label{chap:conc}

\lipsum[4-6]

\section{Trabalho Futuro}\label{sec:trab}

\lipsum[1-3]




%% references
%\renewcommand{\bibname}{Referências} % o babel portuguese coloca Bibliografia
% os meses do ficheiro bib poderão aparecer em inglês, caso se pretenda deve-se colocar o texto em português explicitamente no ficheiro bid
\cleardoublepage%
\phantomsection%
\printbibliography[heading=bibintoc]%


%% appendix
\appendix
\include{appendix1}

%% bye
\end{document}
