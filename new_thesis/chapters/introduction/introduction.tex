\chapter{Introduction}\label{chap:intro}

%https://www.usenix.org/system/files/sec22-turonova.pdf
%https://arxiv.org/abs/2407.20479
%https://www.regular-expressions.info/catastrophic.html

In this chapter, the problem is overviewed, the study's importance is explained along with goals for the proposed solution. 

\section{Background}
Regular expressions are a foundational tool in computer science, widely used in pattern matching, lexical analysis, input validation, and string processing. Their expressiveness and concise syntax make them a powerful language for describing regular languages.

% However, when implemented without care, especially in backtracking-based engines, regular expressions can become a source of serious security vulnerabilities.

A regular expression $R$ is used (along with an input $W$) in regex matching engines. The matching engines will verify if $W$ is fully matched by $R$, meaning that the entire input is a match - or they will verify if a substring of $W$ is matched by $R$. 

% Current matching engines commonly fall into one of the following categories:
% \begin{itemize}
% 	\item \textbf{Finite State Machine Regular Expression Engines}: A finite state machine (or finite \emph{automaton}) is built and evaluated using every symbol $\sigma \in W$. Used \emph{UNIX}-based systems.
% 	\item \textbf{Backtracking Regular Expression Engines}: Instead of building a finite state machine, 
% \end{itemize}

\section{Regular Expression Denial of Service}
One such vulnerability is known as \emph{Regular Expression Denial of Service} (ReDoS). ReDoS exploits the pathological worst-case behavior of certain regular expressions, causing exponential time complexity during matching. In typical backtracking matchers---such as those found in JavaScript, Java, and many scripting environments---ambiguous or nested expressions (especially involving repetition, such as \texttt{(a+)+}) can lead the engine to explore an exponential number of paths for certain crafted inputs. This behavior allows an attacker to intentionally supply inputs that force excessive computation, effectively rendering a service unavailable or degraded.

The root of the ReDoS problem lies not in regular expressions as a theoretical model but in how they are operationalized in software. While deterministic finite automata (DFAs) evaluate regular expressions in linear time, many real-world engines opt for backtracking NFAs due to their flexibility and ease of implementation. Unfortunately, these NFAs are susceptible to exponential blow-up in ambiguous or unguarded patterns.