\chapter{Preliminaries}\label{chap:prelim}
Theory builds upon theory, therefore it is essential to establish a solid foundation by understanding the basic concepts and terminology that compose the core topics of formal languages and automata theory.
In this chapter we begin by formally defining what a language is and then move on to describe the class of languages known as regular languages.
Along the way, we will also introduce various concepts such as finite automata (DFA, NFA) and regular expressions.

\section{Alphabets, Strings and Languages}
\subsection*{Alphabets}

An \emph{alphabet} is a finite, non-empty set of symbols, typically denoted by the Greek letter $\Sigma$. That is,
\[
\Sigma = \{ a_1, a_2, \dots, a_n \}
\]

\noindent where each $a_i$ is a symbol in the alphabet.

For example, one can represent the binary alphabet as $\Sigma = \{ 0, 1 \}$, or the English alphabet as $\Sigma = \{ a, b, c, \ldots, z \}$.

\subsection*{Strings}
A \emph{string} over an alphabet $\Sigma$ is a finite sequence of symbols from $\Sigma$. Strings are typically denoted by $w$, and the \emph{length} of a string $w$ is denoted by $|w|$.

The set of all strings over the alphabet $\Sigma$ is denoted by $\Sigma^*$ and defined as:
\[
\Sigma^* = \{ w \mid w \text{ is a finite sequence of symbols from } \Sigma \}
\]

The unique string of length zero is called the \emph{empty string}, denoted by $\varepsilon$.
It is important to note that $\varepsilon \in \Sigma^*$.

For example, if $\Sigma = \{ 0, 1 \}$, then we have that:
\begin{center}
	$\Sigma^* = \{ \varepsilon, 0, 1, 00, 01, 10, 11, 000, 001, 010, 011, 100, 101, 110, 111, \ldots \}$
\end{center}

Where the empty string is, as mentioned above, denoted by $\varepsilon$ and also belongs to $\Sigma^*$.

\subsection*{Languages}

A \emph{language} over an alphabet $\Sigma$ is a set of strings over $\Sigma$.

\[
L \subseteq \Sigma^*
\]

That is, a language is any subset of $\Sigma^*$, possibly infinite, finite, or even empty. \newline
Since a language is a set of strings, the following standard set operations can be applied (assuming $A$ and $B$ are languages over the same alphabet $\Sigma$):

\begin{itemize}
	\item \emph{Intersection}: $A \cap B = \{ x \mid x \in A \text{ and } x \in B \}$
	\item \emph{Union}: $A \cup B = \{ x \mid x \in A \text{ or } x \in B \}$
	\item \emph{Difference}: $A - B = \{ x \mid x \in A \text{ and } x \notin B \}$
\end{itemize}

Furthermore, we can also operate specifically over languages with the following operations (assuming $L_1$ and $L_2$ are languages over the same alphabet $\Sigma$):

\begin{itemize}
	\item \emph{Concatenation}: $L_1 \cdot L_2 = \{ xy \mid x \in L_1 \text{ and } y \in L_2 \}$
	\item \emph{Kleene Star}: $L^* = \bigcup_{n=0}^{\infty} L^n$, where $L^0 = \{\varepsilon\}$ and $L^n = L \cdot L^{n-1}$ for $n > 0$.
	\item \emph{Reversal}: $L^R = \{ x^R \mid x \in L \}$, where $x^R$ denotes the reversal of string $x$.
	\item \emph{Complement}: $\overline{L} = \Sigma^* - L$, i.e., the set of all strings over $\Sigma$ that are not in $L$.
\end{itemize}

These operations form the basis for reasoning about the expressiveness and closure properties of language classes such as regular, context-free, and context-sensitive languages. In particular, regular languages are closed under all the operations listed above, including union, intersection, concatenation, and Kleene star. This robustness makes them especially amenable to algorithmic manipulation, as seen in finite automata and regular expression engines.


\section{Regular Expressions}
Let $\Sigma$ be a finite alphabet. Let $L \subseteq \Sigma$. The set of \emph{regular expressions} over $\Sigma$, denoted by $\text{RegExp}(\Sigma)$, is defined inductively as follows:

\begin{itemize}
    \item $\emptyset$ is a regular expression denoting the empty language: $L(\emptyset) = \emptyset$.
    \item $\varepsilon$ is a regular expression denoting the language containing only the empty string: $L(\varepsilon) = \{ \varepsilon \}$.
    \item For each symbol $a \in \Sigma$, $a$ is a regular expression denoting the singleton language: $L(a) = \{ a \}$.
    \item If $r_1$ and $r_2$ are regular expressions, then so are:
    \begin{itemize}
        \item $(r_1 \mid r_2)$, denoting the union: $L(r_1 \mid r_2) = L(r_1) \cup L(r_2)$.
        \item $(r_1 \cdot r_2)$, denoting concatenation: $L(r_1 \cdot r_2) = L(r_1) \cdot L(r_2)$.
        \item $(r_1)^*$, denoting Kleene star: $L(r_1^*) = (L(r_1))^*$.
    \end{itemize}
\end{itemize}
We write $\text{RegExp}(\Sigma)$ to denote the set of all such syntactic expressions, and for each $r \in \text{RegExp}(\Sigma)$, the function $L(r)$ yields the language defined by $r$.

\medskip

Parentheses are used to disambiguate expressions and enforce precedence; by convention, Kleene star binds most tightly, followed by concatenation, and finally union.

\subsection{Extended Regular Expressions}
In addition to the basic operations, some extended operators are often used for convenience. These include:

\begin{itemize}
    \item \textbf{Kleene plus}: Given a regular expression $r$, the expression $r^+$ denotes one or more repetitions of $r$:
    \[
    L(r^+) = L(r) \cdot L(r)^*.
    \]

    \item \textbf{Fixed repetition (power)}: For a regular expression $r$ and integer $n \geq 0$, the expression $r^n$ denotes $n$ consecutive concatenations of $r$:
    \[
    L(r^0) = \{ \varepsilon \}, \quad L(r^n) = L(r) \cdot L(r^{n-1}) \text{ for } n > 0.
    \]

	 \item \textbf{Bounded repetition}: For a regular expression $r$ and integers $m, n$ with $0 \leq m < n$, the bounded repetition $r^{[m,n[}$ denotes the language containing all strings formed by concatenating between $m$ and $n$ copies of strings from $L(r)$:
    \[
    L(r^{[m,n[}) = \bigcup_{k=m}^{n} L(r^k).
    \]
\end{itemize}

These extended forms do not increase the expressive power of regular expressions but are useful for readability and practical applications. They can always be rewritten using the fundamental operators: union and concatenation.

\subsection{Finite Automata}

A \emph{finite automaton} is a theoretical machine used to recognize regular languages. It processes input strings symbol by symbol and determines whether the string belongs to the language defined by the automaton. There are two main types of finite automata:

\begin{itemize}
    \item \textbf{Deterministic Finite Automaton (DFA)}: An automaton where, for each state and input symbol, there is exactly one possible next state.
    \item \textbf{Nondeterministic Finite Automaton (NFA)}: An automaton that allows multiple possible transitions for a given state and input symbol, including transitions without consuming any input (called $\varepsilon$-transitions).
\end{itemize}

\subsubsection*{Formal Definition of an NFA}

An NFA is a 5-tuple $(Q, \Sigma, \delta, q_0, F)$, where:

\begin{itemize}
    \item $Q$ is a finite set of states,
    \item $\Sigma$ is the input alphabet,
    \item $\delta: Q \times (\Sigma \cup \{\varepsilon\}) \rightarrow 2^Q$ is the transition function,
    \item $q_0 \in Q$ is the initial state,
    \item $F \subseteq Q$ is the set of accepting (final) states.
\end{itemize}

A string $w \in \Sigma^*$ is accepted by the NFA if there exists a sequence of transitions (possibly including $\varepsilon$-moves) that consumes $w$ and ends in a state $q$ such that $q \in F$.

\subsection{Position Automata}
%https://www.dcc.fc.up.pt/~nam/resources/publica/dlt16.pdf

To first recall the notion of \emph{position automata} (also known as \emph{Glushkov automata}), one must first understand the idea of marking each occurrence of a symbol in a regular expression with a unique position. This transforms a regular expression into a \emph{marked regular expression}, where each letter is indexed according to its position in the expression when read left to right. The position automaton is then constructed from this marked version by associating each position with a distinct state, and defining transitions based on the structural analysis of the expression. \cite{mesh-of-automata}

The key to this construction lies in the inductive computation of three sets: \emph{First}, \emph{Last}, and \emph{Follow}.The \emph{First} set contains positions that may begin a word in the language; the \emph{Last} set includes positions that may end such words; and the \emph{Follow} relation determines which positions can immediately follow a given position in any word generated by the expression. The resulting automaton has states corresponding to positions, transitions derived from the \emph{Follow} relation, an initial state representing position 0 (the start of matching), and final states determined by the \emph{Last} set, extended with position 0 if the empty string is accepted. \cite{mesh-of-automata}

\subsection{Derivatives}
% Need 1962 Janusz Brzozowski's paper for this section
The \emph{derivative of a regular expression} was first introduced in 1962 by Janusz Brzozowski. It is a powerful concept used to define the behavior of regular expressions in a more operational manner. The derivative of a regular expression $r$ with respect to a symbol $a$ is another regular expression $D_a(r)$ that describes the set of strings that can be obtained by taking the derivative of $r$ with respect to $a$.

The derivative can be defined based on the structure of the regular expression:

\begin{itemize}
    \item If $r = \varepsilon$, then $D_a(r) = \emptyset$ for all $a \in \Sigma$.
    \item If $r = a$, then $D_a(r) = \varepsilon$.
    \item If $r = r_1 \cdot r_2$, then $D_a(r) = D_a(r_1) \cdot L(r_2) \cup \varepsilon \cdot D_a(r_2)$.
    \item If $r = r_1 + r_2$, then $D_a(r) = D_a(r_1) + D_a(r_2)$.
    \item If $r = r^*$, then $D_a(r) = D_a(r) \cdot r^*$.
\end{itemize}