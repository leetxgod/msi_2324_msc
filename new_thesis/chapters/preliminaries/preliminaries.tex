\chapter{Preliminaries}\label{chap:prelim}
Theory builds upon theory, therefore it is essential to establish a solid foundation by understanding the basic concepts and terminology that compose the core topics of formal languages and automata theory.
In this chapter we begin by formally defining what a language is and then move on to describe the class of languages known as regular languages.
Along the way, we will also introduce various concepts such as finite/non-finite automata and regular expressions.

\section{Alphabets, Strings and Languages}
\subsection*{Alphabets}

An \emph{alphabet} is a finite, non-empty set of symbols. It is usually denoted by the Greek letter $\Sigma$.

\[
\Sigma = \{ a_1, a_2, \dots, a_n \}
\]

where each $a_i$ is a symbol. \newline

For example, one can represent the binary alphabet as $\Sigma = \{ 0, 1 \}$, or the English alphabet as $\Sigma = \{ a, b, c, \ldots, z \}$.

\subsection*{Strings}

A \emph{string} over an alphabet $\Sigma$ is a finite sequence of symbols from $\Sigma$.

- The empty string (the string of length 0) is denoted by $\varepsilon$.
- If $w$ is a string, then $|w|$ denotes the \emph{length} of $w$.
- The set of all strings (including the empty string) that can be formed from $\Sigma$ is denoted by $\Sigma^*$.

\[
\Sigma^* = \{ w \mid w \text{ is a finite sequence of symbols from } \Sigma \}
\]

For example, if $\Sigma = \{ 0, 1 \}$, then we have that:
\begin{center}
	$\Sigma^* = \{ \varepsilon, 0, 1, 00, 01, 10, 11, 000, 001, 010, 011, 100, 101, 110, 111, \ldots \}$
\end{center}

Where the empty string is, as mentioned above, denoted by $\varepsilon$ and also belongs to $\Sigma^*$.

\subsection*{Languages}

A \emph{language} over an alphabet $\Sigma$ is a set of strings over $\Sigma$.

\[
L \subseteq \Sigma^*
\]

That is, a language is any subset of $\Sigma^*$, possibly infinite, finite, or even empty. \newline
Since a language is a set of strings, the following standard set operations can be applied: % such as \emph{union}, \emph{intersection}, and \emph{complement} can be applied to languages.
\begin{itemize}
	\item \emph{Intersection}: $A \cap B = \{ x \mid x \in A \text{ and } x \in B \}$
	\item \emph{Union}: $A \cup B = \{ x \mid x \in A \text{ or } x \in B \}$
	\item \emph{Difference}: $A - B = \{ x \mid x \in A \text{ and } x \notin B \}$
	% \item \emph{Union}: $L_1 \cup L_2 = \{ w \mid w \in L_1 \text{ or } w \in L_2 \}$
	% \item \emph{Intersection}: $L_1 \cap L_2 = \{ w \mid w \in L_1 \text{ and } w \in L_2 \}$
\end{itemize}

Furthermore, we can also operate 


The \emph{complement} of a language $L$ over an alphabet $\Sigma$ is denoted by $\overline{L}$ and is defined as:

